\documentclass[../main.tex]{subfiles}

\begin{document}

\setcounter{chapter}{3}

\chapter{Schemes and algebraic spaces I}

In the previous lecture we saw the notion of a geometric context, and that this notion gives rise to manifolds and geometric spaces. We will now construct a context specific to algebraic geometry in which the manifolds are schemes and geometric spaces are algebraic spaces.


In order to understand the conrtuction of algebraic contexts we commence by recalling some elementary facts about (affine) algebraic varieties, .... Naively an algebraic variety is a set of solutions to a system of plynomial equations. ... An algebraic variety is determined a finite familyof polynomials $P_1, \cdot P_r \in \mathbb C[x_1, \cdots, x_n]$. X is then a solution a dset of solutions of the system $\{P_i(x) = 0\}_{1 \le i \le r}$, in ... terms we have

\[
X = \{ x \in \mathbb C^n ~|~ P_i(x) = 0 ~ \forall i\}
\]
We also then have the question of defining the ``algebraic functions'' on $X$. ... ring of algebraic functions on $X$ is witten as

\[
\mathcal O(X) = \mathcal C[x_1, \cdots x_n] / (P_1, \cdots, P_r)
\]

An algebraic variety $X$ corresponds to the ring $C[x_1, \cdots x_n] / (P_1, \cdots, P_r)$. This ring is a commutative $\mathbb C$-algebra of finite type (i.e. with a finite number of generators). Conversely, if $A$ is an commutative $C$-algebra of finite type, we can write

\[
A \simeq \mathbb C[y_1, \cdots, y_m]/(Q_1, \cdots, Q_s)
\]

and construct an algebraic variety in $\mathbb C^m$ with the equation $Q_1(y) = \cdots = Q_s(y) = 0$.


there's a correspondence between affines algebraic varies in $\mathbb C$ (i.e. subvarieties of $\mathbb C^n$ for some $n$), and commutative $\mathbb C$-algebras of finite type. The starting point for the theory of schemes and considering commutative rings A (not necessarily commutative $\mathbb C$-algebras of finite type) is the considering of the correspondenc between a geometric object $X$ and $A$ a ring of algebraic functions on $X$. The geometric objects associated with the ring are affine schemes, and a scheme is by definition a collection of affine schemes (much like a topological property is a collection of opens in $\mathbb R^n$). We have the following two principles

\begin{prcp*}
    Affine schemes are in correspondence with commutative rings
\end{prcp*}
\begin{prcp*}
    Schemes are obtained from collections of affines schemes
\end{prcp*}


From the combination two principles we obtain the following principles

\begin{prcp*}
    Schemes are obtained from collections of commutative rings
\end{prcp*}



\section{Geometric contexts}




\end{document}