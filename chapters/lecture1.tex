\documentclass[../main.tex]{subfiles}

\begin{document}

\setcounter{chapter}{0}
\chapter{Reflections on the Notion of Space I}

The purpose of these first lectures is to understand the notion of a manifold in different contexts (topological/differential/analytic manifolds). We begin in this first lecture by looking at the case of topological manifolds.

\section{Reminders on topological manifolds}

\begin{defn}


\begin{enumerate}
    \item A \textbf{topological manifold} is a topological space $X$ that has an open cover $\{U_i\}_{i \in I}$, such that for each $i \in I$, there exists a homeomorphism from $U_i$ to an open set of $\mathbb R^{n_i}$ (for some $n_i\ge 0$ depedent on $i$).
    \item The category of topological manifolds is the full subcategory of topological spaces whose objects are topological manifolds. It is denoted by $\mathbf{TopMfd}$.
\end{enumerate}

\end{defn}

Let $X$ be a topological manifold and $\{U_i\}_{i \in I}$ an open cover as in definition $1.1 (1)$. We write, for $i$ and $j$ in $I$, $U_{i,j} = U_n \cap U_j$. There's a diagram of topological spaces

\[
    \bigsqcup_{(i, j) \in I^2} U_{i, j} \rightrightarrows \bigsqcup_{i \in I} U_i
\]

The first morphism sends $U_{i, j}$ to $U_i$ via the natural inclusion $U_{i,j}
subset U_i$, and the second morphism sends $U_{i, j}$ to $U_j$ via the natural inclusion $U_{i, j}
\subset U_j$. There's also another morphism

\[
    \bigsqcup_{i \in I} U_i \to X
\]

which are inclusions $U_i \subset X$. These coequalizes the two morphisms above. So we get a morphism of topological spaces

\[
Colim\left(\bigsqcup_{(i, j) \in I^2} U_{i, j} \rightrightarrows \bigsqcup_{i \in I} U_i\right) \to X
\]

The important fact is the following

\begin{lem}

The morphism

\[
Colim\left(\bigsqcup_{(i, j) \in I^2} U_{i, j} \rightrightarrows \bigsqcup_{i \in I} U_i\right) \to X
\]

is an isomorphism.
\end{lem}

\begin{proof}
The lemma says for a topological space $Y$, to give a morphism $f: X \to Y$ is the same as choosing, for each given $i \in I$, a morphism $f_i: U_i \to Y$ such that $(f_i)|_{U_{i, j}} = (f_j)|_{U_{i, j}}$ for all $(i, j) \in I^2$.

(Exercise: provide the details.)
\end{proof}

One can interpret the above lemma as follows: all topological manifolds are obtains as the colimit of a diagram of opens in $\mathbb R^n$ (for some $n$). We draw from this the following principle:

\begin{prcp*}
The category \textbf{TopMfld} of topological manifolds is.
\end{prcp*}
\section{Manifolds and sheaves}

\section{Quotient manifolds}

\section{Shortcomings of manifolds}

\end{document}