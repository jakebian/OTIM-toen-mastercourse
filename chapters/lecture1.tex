\documentclass[../main.tex]{subfiles}

\begin{document}

\setcounter{chapter}{0}
\chapter{Reflections on the Notion of Space I}

The purpose of these first lectures is to understand the notion of a manifold in different contexts (topological/differential/analytic manifolds). We begin in this first lecture by looking at the case of topological manifolds.

\section{Reminders on topological manifolds}

\begin{defn}


\begin{enumerate}
    \item A \textbf{topological manifold} is a topological space $X$ that has an open cover $\{U_i\}_{i \in I}$, such that for each $i \in I$, there exists a homeomorphism from $U_i$ to an open set of $\mathbb R^{n_i}$ (for some $n_i\ge 0$ depedent on $i$).
    \item The category of topological manifolds is the full subcategory of topological spaces whose objects are topological manifolds. It is denoted by $\mathbf{TopMfd}$.
\end{enumerate}

\end{defn}

Let $X$ be a topological manifold and $\{U_i\}_{i \in I}$ an open cover as in definition $1.1 (1)$. We write, for $i$ and $j$ in $I$, $U_{i,j} = U_n \cap U_j$. There's a diagram of topological spaces

\[
    \bigsqcup_{(i, j) \in I^2} U_{i, j} \rightrightarrows \bigsqcup_{i \in I} U_i
\]

The first morphism sends $U_{i, j}$ to $U_i$ via the natural inclusion $U_{i,j}
subset U_i$, and the second morphism sends $U_{i, j}$ to $U_j$ via the natural inclusion $U_{i, j}
\subset U_j$. There's also another morphism

\[
    \bigsqcup_{i \in I} U_i \to X
\]

which are inclusions $U_i \subset X$. This coequalizes the two morphisms above. So we get a morphism of topological spaces

\[
Colim\left(\bigsqcup_{(i, j) \in I^2} U_{i, j} \rightrightarrows \bigsqcup_{i \in I} U_i\right) \to X
\]

The important fact is the following

\begin{lem}

The morphism

\[
Colim\left(\bigsqcup_{(i, j) \in I^2} U_{i, j} \rightrightarrows \bigsqcup_{i \in I} U_i\right) \to X
\]

is an isomorphism.
\end{lem}

\begin{proof}
The lemma says for a topological space $Y$, to give a morphism $f: X \to Y$ is the same as choosing, for each given $i \in I$, a morphism $f_i: U_i \to Y$ such that $(f_i)|_{U_{i, j}} = (f_j)|_{U_{i, j}}$ for all $(i, j) \in I^2$.

(Exercise: provide the details.)
\end{proof}

One can interpret the above lemma as follows: all topological manifolds are obtains as the colimit of a diagram of opens in $\mathbb R^n$ (for some $n$). We draw from this the following principle:

\begin{prcp*}
The category $TopMfd$ of topological manifolds is deduced from the category of opens in $\mathbb R^n$ (and continiuous maps.).
\end{prcp*}

this is the principle that we're going to clarify in the following.

\section{Manifolds and sheaves}

Let $C$ be the full subcategory of $TopMfd$ whose objects are opens of $\mathbb R^n$. We denote by $Pr(C)$ the category of presheaves of sets over $C$ (also denoted $\hat C$). We consider the Yoneda embedding restricted to $C$:

\begin{align}
h_{(-)}: VarTop &\to Pr(C)\\
X &\mapsto h_X
\end{align}

where the presheaf $h_X$ is defined by

\[
h_X(Y) := Hom_{TopMfd}(Y, X)
\]


\begin{lem}
    The functor $h_{(-)}$ above is fully faithful.
\end{lem}

\begin{proof}
The functor is faithful: ...

The functor is full: ...

TODO: finish
\end{proof}

The lemma 2.1 is a good point of depature, for $TopMfd$ is identified as (is equivalent to) a full subcategory of $Pr(C)$. We seek a characterization of this subcategory.

We start with $C$ a Grothendieck site by declaring a family of morphisms $\{U_i \to U\}_{i \in I}$ in C to be a covering family if each morphism $U_i \to U$ is an open immersion, and the total morphism $\sqcup_{i \in I} U_i \to U$ is surjective. This defines a pre-topology on $C$ (exercise: verify). The associated topology is denoted $\tau$.

\begin{lem}
    For all $X \in TopMfd$, the presheaf $h_X \in Pr(C)$ is a sheaf for the topology $\tau$.
\end{lem}

\begin{proof}
    ...
\end{proof}

Lemma 2.2 implies we have a full and faithful functor

\[
h_{(-)}: TopMfld \to Sh(C, \tau)
\]

A sheaf isomorphic to $h_X$ is said to be representable by $X$. More generally we identify the category $TopMfd$ with its image in $Sh(C, \tau)$

To chacaterize this image we make a definition

\begin{defn}
    \begin{enumerate}
            \item A morphism $f: F \to G$ in $Sh(C, \tau)$ is a \textbf{local homeomphism} if for all $X \in C$, and all morphisms $h_X \to G$, the sheaf $F \times_G h_X$ is representable by some $Y \in TopMfd$, and the induced morphism $Y \to X$ by the projection $h_y \simeq F \times_G h_X \to h_X$ is a local homeomorphism of topological spaces \footnote{Recall: a continuous map of topological spaces is a local homeomorphism if for each $x \in X$, there exists $U$ an open neighborhood of $x$ in $X$ and $V$ and open neighborhood of $f(x)$ in Y, such that $f$ induces a homeomorphism from $U \to V$}.

            \item A morphism in $Sh(C, \tau)$ is an \textbf{open immersion} if it's a monomorphism and a local homeomorphism.
    \end{enumerate}
\end{defn}

It's easy to verify that open immersions in $Sh(C, \tau)$ are stable under compositions (Exercise: verify). One can also verify that local heomorphisms are stable under compositions, but this needs corrolary 2.5 below (Exercise: verify). We see also that a morphism of topological manifolds is a local homeomorphism if and only if $h_X \to h_Y$ is a local homeomorphism in the sense of the above definition (Exercise: verify).

We then have the following proposition

\begin{prop}
    A sheaf $F \in Sh(C, \tau)$ is representable by a topological manifold (i.e. $F \simeq h_X$ for some $X \in TopMfd$), if there exists a family $\{U_i\}_{i \in I}$ of objects in $C$, and a morphism of sheaves

    \[
        p: \bigsqcup_{i \in I} h_{U_i} \to F
    \]
    satisfying the following two conditions

    \begin{enumerate}
        \item The morphism $p$ is an epimorphism of sheaves.
        \item For all $i \in I$, the morphism $U_i \to F$ is an open immersion (in the sense of definition 2.3)
    \end{enumerate}
\end{prop}

\begin{proof}
\end{proof}

\begin{cor}
    Let $X \in TopMfd$, and $F \to h_X$ a morphism of sheaves. If there exists an open cover of $X$ such that for all $i \in I$, the sheaf $f \times_{h_X} h_{U_i}$ is representable by a topological manifold, then $F$ is representable by a topological manifold.
\end{cor}
\begin{proof}
    For all $i \in I$, choose ${V_{i, j}}_{j \in J}$...
\end{proof}


\section{Quotient manifolds}

Let G be a (discrete) group acting on a topological manifold $X \in TopMfd$. By functoriality, the group $G$ acts on the sheaf $h_X$. Recall a group action on X is ... . Recall also a group action $G$ on $X$ is properly discontinuous if all points $x \in X$ has an open neighborhood $U \subset X$ such that for all $g \in G$, we have

\[
    g(U) \cap U \ne \emptyset \implies (g = e)
\]

In the following, we will take care not to confuse the sheaf quotient $h_X/G$ and the sheaf $h_{X/G}$ represented by the quotient topological space.

\begin{prop}
    \begin{enumerate}
            \item If the action $G$ on $X$ is free, the the quotient morphism
            \[h_X \to h_X/G\] is a local homeomorphism
            \item If the action $G$ on $X$ is properly discontinuous, then the quotient $h_X/G \in Sh(C, \tau)$ is a topological manifold.
    \end{enumerate}
\end{prop}
\begin{proof}
...
\end{proof}

\section{Shortcomings of manifolds}

The proposition 3.1 is a good reason for cosntructing examples of topological manifolds by properly discontinuous actions. However, when $G$ acts on a manifold $X$ but the action is not propertly discontinuous, the topological space quotient $X/G$ is in general very pathological. The sheaf quotient $h_X/G$. The sheaf quotient $h_X/G$ has good properties (e.g. point (1) of proposition 3.1) similar to representability of a topological manifold.

An example is the following: we take the action of the discrete group $\mathbb Q$ (under addition on the topological space $\mathbb R$ via the morphism

\[\mathbb R \times \mathbb Q \to \mathbb R\]

given by $(x, t) \mapsto x + t$. We note this action is free, but not properly discontinuous. IN addition, the morphism $\mathbb R \to \mathbb R / \mathbb Q$ is not a local homeomorphism nor is it locally injective. Finally, the topological space quotient $\mathbb R / \mathbb Q$ has a gross topology. We see that the quotient $\mathbb R/ \mathbb Q$ is not a reasonable object from the point of view of geometry. On the otherhand, the sheaf quotient $h_\mathbb R/\mathbb Q$ is more interesting, for the morphisms $h_\mathbb R \to h_\mathbb R / \mathbb Q$ is a local homeomorphism. The sheaf $h_\mathbb R / \mathbb Q$ is a primary example of a geometric space

\begin{defn}
    A sheaf $F \in Sh(C, \tau)$ is a geometric space if there exists a family of objects $\{U_i\}_{i \in I}$ of $C$, and a morphism of sheaves

    \[
        p: \bigsqcup_{i \in I} h_{U_i} \to F
    \]
    satisfies the following two conditions

    \begin{enumerate}
        \item The morphism $p$ is an epimorphism of sheaves
        \item For all $i\in I$, the morphism $U_i \to F$ is a local homeomorphism.
    \end{enumerate}
\end{defn}

\end{document}