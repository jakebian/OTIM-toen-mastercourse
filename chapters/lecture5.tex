\documentclass[../main.tex]{subfiles}

\begin{document}

\setcounter{chapter}{5}

\chapter{Stacks I}

We recall that for a category $C$ and a subset $W \subset C_1$ of morphisms in $C$, a localization of $C$ along $W$ is a category $W^{-1}C$ with a functor $l: C \to W^{-1}C$ such that for all other categories $D$, the functor

\[
l^*: \underline{Hom}(W^{-1}C, D),  \underline{Hom}(C, D)
\]

is full and faithful, and the essential image consists of functors $C \to D$ sending $W$ to isomorphisms in $D$. If a localization exists it's unique up to isomorphism.


\section{Homotopy theory of groupoids}

We consider the case where $C = Gpd$, the category of groupoids (i.e. objects are groupoids and morphisms are functors of groupoids). We take $W$ the subset of equivalence of groupoids (i.e. the functors that are equivalences of category) and we we consider the category $W^{-1}Gpd$.

\begin{defn}
    The homotopy category of groupoids is the localization $W^{-1}Gpd$. We also write $Ho(Gpd)$. The set of morphisms in $Ho(Gpd)$ between objects $A$ and $B$ is denoted $[A, B]$.
\end{defn}

We now denote [Gpd] the category where objects are groupoids, and for groupoids $A$ and $B$, the sect of morphisms between $A$ and $B$ in $[Gpd[$ is by definition the set of isomorphism classes of functors from $A$ to $B$. There's a natural projection $p: Gpd \to [Gpd]$, which is the identity on objects and the canonical projection on morphism sets. Exercise: write down compositions in $[Gpd]$


\begin{thm}
    The natural project
    \[
        p: Gpd \to [Gpd]
    \]
    is a localization of $Gpd$ along $W$. It follows that $p$ induces a natural equivalence

    \[
        Ho(Gpd) \simeq [Gpd]
    \]

\end{thm}

\begin{proof}
...
\end{proof}

\begin{cor}
    The natural functor
    \[
        j: Set \to Gpd \to Ho(Gpd)
    \]
    sending a set to the corresponding discrete groupoid, is full and faithful, and has a left adjoint:
    \[
        \pi_0: Ho(Gpd) \to Set
    \]
\end{cor}

\begin{proof}
...
\end{proof}

Exercise: Let $G$ and $H$ be groups. We write $BG$ (resp. $BH$) the groupoids with a single object and $G$ (and $H$) group of automorphisms of the object. Describe the set $[BG, BH]$ of morphisms in the category Ho(Gpd).

\section{Homotopy theory of diagrams of groupoids}

Let $I$ be a category, and consider $\underline{Hom}(I, Gpd)$ the category of functors from $I$ to $Gpd$ (also called the category of $I$-diagrams in $Gpd$). Take $F, G: I \to Gpd$ two $I$-diagrams. A morphism
\[
    f: F \to G
\]
is a equivalence if on each object $i\in I$, the induced morphism

\[
f_i: F(i) \to G(i)
\]
is an equivalence of groupoids. This defines the class $W_I$ of weak equivalences in $\underline{Hom}(I, Gpd)$.
\begin{defn}
    The homotopy category of $I$-diagrams of groupoids is $W_I^{-1}\underline{Hom}(I, Gpd)$. Also denoted $Ho(\underline{Hom}(I, Gpd))$. The set of morphisms in $W_I^{-1}\underline{Hom}(I, Gpd)$ from $F$ to $G$ is denoted $[F, G]$.
\end{defn}

To generalization of theorem 1.2. to the case of digrams of groupoids, we need to introduce the notion of weak morphisms between objects in $\underline{Hom}(I, Gpd)$. For this, we denote, for $F \in \underline{Hom}(I, Gpd)$, and for morphisms $u: i \to j$ in $I$, $u_*^F: F(i) \to F(j)$ the functor induced by $u$. Note by definition, $(u^f \circ v^f)_* = u_*^F \circ v_*^f$ and $id_*^F = id$.

\begin{thm}
    The natural functor
    \[
        \underline{Hom}(I, Gpd) \to [\underline{Hom}(I, Gpd)]
    \]
    induces an equivalence
    \[
        Ho(\underline{Hom}(I, Gpd) \simeq [\underline{Hom}(I, Gpd)]
    \]

\end{thm}



\section{Homotopy limits}
\section{Homotopy categories of prestacks and stacks}

\end{document}