\documentclass[../main.tex]{subfiles}

\begin{document}

\setcounter{chapter}{0}
\chapter{Reflections on the Notion of Space I WIRJ}

The purpose of these first lectures is to understand the notion of a manifold in different contexts (topological/differential/analytic manifolds). We begin in this first lecture by looking at the case of topological manifolds.

\section{Reminders on topological manifolds}

\begin{defn}


\begin{enumerate}
    \item A \textbf{topological manifold} is a topological space $X$ that has an open cover $\{U_i\}_{i \in I}$, such that for each $i \in I$, there exists a homeomorphism from $U_i$ to an open set of $\mathbb R^{n_i}$ (for some $n_i\ge 0$ depedent on $i$).
    \item The category of topological manifolds is the full subcategory of topological spaces whose objects are topological manifolds. It is denoted by $\mathbf{TopMfd}$.
\end{enumerate}

\end{defn}

Let $X$ be a topological manifold and $\{U_i\}_{i \in I}$ an open cover as in definition $1.1 (1)$. We write, for $i$ and $j$ in $I$, $U_{i,j} = U_n \cap U_j$.

\section{Manifolds and sheaves}

\section{Quotient manifolds}

\section{Shortcomings of manifolds}

\end{document}